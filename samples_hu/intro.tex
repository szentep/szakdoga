\chapter{Bevezetés}
\label{ch:intro}

\section{Témaválasztás és feladatleírás}

A számítógépes modellezés egyik meghatározó eszköze a testek felületének különböző parametrikus felületekkel való leírása. Egyre szélesebb körben elterjedő implicit felületreprezentáció az előjeles távolságfüggvények és diszkretizált változatuk, a távolságmezők. A távolságmezővel reprezentált felület egy speciális sugárkövető eljárással, a sphere tracinggel jeleníthető meg.

A szakdolgozat célja parametrikus felületekkel ábrázolt objektumok távolságmezőjének generálására GPU segítségével. A létrejövő távolságmezők megjeleníthetők GPU-val gyorsított módon, az említett sphere tracing módszerrel. Ezt a módszert összehasonlítom a parametrikus felület háromszögekkel tesszellált megjelenítésével.

A távolságmezőt felhasználom még geometriai lekérdezések elvégzésére, mint a felületi normális meghatározása, ami a felület árnyalásához elengedhetetlen. Meghatározhatók még láthatósági viszonyok, melyet vetett árnyékok számításánál használok.

A készített programmal kétdimenziós függvények grafikonjai és Bézier--felületek jeleníthetők meg. A Bézier-felületekre azért esett a választás, mert a kontrollpont-hálójukon keresztül intuitívan manipulálhatók és a folytonosság megtartása mellett összeilleszthetők, így különösen alkalmasak felületek modellezésére. Harmadfokú Bézier--felületek távolságmezőjének generálására speciális algoritmust is adok, melyet összehasonlítok a korábban tárgyalt módszerekkel.

\section{Szakirodalmi áttekintés}

A téma \citeauthor{Hart1996} cikkével \cite{Hart1996} kezdődik, melyben bemutatja, hogyan használható a sphere trace algoritmus felületek megjelenítésére. A módszer a sugárkövetést gyorsítja fel azzal, hogy a felülettől vett távolsággal lép a sugáron. Ez különösen a felülettől távol eredményez gyorsítást. A módszer előfeltétele, hogy a távolságfüggvényt viszonylag gyorsan ki tudjuk számítani. Ez könnyen megtehető egyszerűbb testek esetében. A módszert általában a GPU pixel shaderében implementálják, így a teljes színtér eltárolható GPU kódként. Emiatt a Demoscene-ekben nagyon sok példát látunk rá. Ilyeneket gyűjt össze például a Shadertoy weboldal.

Ha a felületek távolságfüggvénye nehezen számítható, vagy a jelenet túl összetett, akkor közelítésekre lehet szükség. Ha a távolságfüggvény becslés nem elég jó, akkor a sugárkövetés lelassul. Az alap sugárkövető algoritmus felgyorsítására több módszer is született. \cite{Keinert2014EnhancedST,BálintValasekAccST,QuadricTrace,RobBan} A sugárkövetés gyorsítható, ha a távolságfüggvényt egy diszkrét rácson kiértékeljük és az eredményt letároljuk. Ennek a műveletnek nem kell valós idejűnek lennie, minden objektumra előszámítható. A távolságmezőből a távolságfüggvény becslését úgy kapjuk, hogy a szomszédos 8 pontot trilineárisan interpoláljuk. Ha a távolságmezőben pontos értékek vannak, akkor a 8 távolság által reprezentált harmadfokú felület pontosan rekonstruálható. \cite{Hansson-Soderlund2022SDF}


